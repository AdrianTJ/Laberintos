\documentclass[spanish,twoside]{book}

%%%%%%%%%%%%%%%%%%%%%%%%%%%%%%%%%%%%%%%%%%%%%%%%%%%%%%%%%%%%%%%%%%%%%%%%%%%%%%%%
%%%%%%%%%%%%%%%%%%%%%%%%%%%%%%%%%%%%% PAQUETES %%%%%%%%%%%%%%%%%%%%%%%%%%%%%%%%%
%%%%%%%%%%%%%%%%%%%%%%%%%%%%%%%%%%%%%%%%%%%%%%%%%%%%%%%%%%%%%%%%%%%%%%%%%%%%%%%%

\usepackage{amsthm}
\usepackage{fullpage}
\usepackage{amsmath}
\usepackage{wrapfig}
\usepackage[vcentering,dvips,hcentering]{geometry}
\usepackage{mathrsfs}
%%%%%%%%%%%%%%%%%%%%%%%%%%%%%%%%%%%%%%%%%%%%%%%%%%%%%%%%%%%%%%%%%%%%%%%%%%%%%%%%
\usepackage{quattrocento}
\usepackage[T1]{fontenc}
%%%%%%%%%%%%%%%%%%%%%%%%%%%%%%%%%%%%%%%%%%%%%%%%%%%%%%%%%%%%%%%%%%%%%%%%%%%%%%%%
\usepackage{ulem}
\usepackage{amssymb}
\usepackage[spanish,activeacute]{babel}
\usepackage[utf8]{inputenc}     
\usepackage[pdftex]{color,graphicx}     
\usepackage{here}
\usepackage{graphicx}
\usepackage[usenames,dvipsnames]{xcolor}
\usepackage{fancyhdr}
\usepackage{etoolbox}
\usepackage{float}  % Para poner H en las figuras
\usepackage{url} % Para poner URL's en la bibliografía/pies de nota.
\usepackage{titlesec}
\usepackage{listings}


%%%%%%%%%%%%%%%%%%%%%%%%%%%%%%%%%%%%%%%%%%%%%%%%%%%%%%%%%%%%%%%%%%%%%%%%%%%%%%%%
%%%%%%%%%%%%%%%%%%%%%%%%%%%%% ESTILO DE PÁGINA %%%%%%%%%%%%%%%%%%%%%%%%%%%%%%%%%
%%%%%%%%%%%%%%%%%%%%%%%%%%%%%%%%%%%%%%%%%%%%%%%%%%%%%%%%%%%%%%%%%%%%%%%%%%%%%%%%

\geometry{papersize={168mm,225mm},total={124mm,185mm}}
\addtolength{\topmargin}{-1cm}
\addtolength{\oddsidemargin}{-1.1cm}
\addtolength{\evensidemargin}{-1.1cm}
\addtolength{\textwidth}{.7in}
\addtolength{\headsep}{.4in}
\pagestyle{fancy}
\titleformat{\section}
  {\normalfont\fontsize{12}{15}\bfseries}{\thesection}{1em}{}
\patchcmd{\thebibliography}{\chapter*}{\section*}{}{} % Para que la bibliografía aparezca en la misma página
\addto\captionsspanish{\renewcommand{\bibname}{Referencias}}
\decimalpoint
\newtheorem{theorem}{Teorema}
\newtheorem{definition}{Definición}
\newtheorem*{proposition}{Proposición}
\newtheorem*{corollary}{Corolario}
\newtheorem*{example}{Ejemplo}
\newtheorem*{lemma}{Lema}



%%%%%%%%%%%%%%%%%%%%%%%%%%%%%%%%%%%%%%%%%%%%%%%%%%%%%%%%%%%%%%%%%%%%%%%%%%%%%%%%
%%%%%%%%%%%%%%%%%%%%%%%%%%%%% CONFIGURACIÓN CÓDIGO %%%%%%%%%%%%%%%%%%%%%%%%%%%%%
%%%%%%%%%%%%%%%%%%%%%%%%%%%%%%%%%%%%%%%%%%%%%%%%%%%%%%%%%%%%%%%%%%%%%%%%%%%%%%%%

\lstset
{ %Formatting for code in appendix
    language=Python,
    basicstyle=\footnotesize,
    numbers=left,
    stepnumber=1,
    showstringspaces=false,
    tabsize=1,
    breaklines=true,
    breakatwhitespace=false,
}

% Para insertar código, cambiar el lengiaje a el lenguaje del código presentado. Como default, está Python. 



%%%%%%%%%%%%%%%%%%%%%%%%%%%%%%%%%%%%%%%%%%%%%%%%%%%%%%%%%%%%%%%%%%%%%%%%%%%%%%%%
%%%%%%%%%%%%%%%%%%%%%%%%% INICIALIZACIÓN DE DOCUMENTO %%%%%%%%%%%%%%%%%%%%%%%%%%
%%%%%%%%%%%%%%%%%%%%%%%%%%%%%%%%%%%%%%%%%%%%%%%%%%%%%%%%%%%%%%%%%%%%%%%%%%%%%%%%

\begin{document}
\setcounter{page}{1} %La página en la que empieza el artículo
\fancyhead{}
\renewcommand{\headrulewidth}{2pt}% 2pt header rule
\renewcommand{\headrule}{\hbox to\headwidth{%
  \color{Purple}\leaders\hrule height \headrulewidth\hfill}}%color de la línea, BurntOrange para Matematico del #, NavyBlue para Axiomas Teoremas, Green para Aterrizando, Red para Activa, Goldenrod para Olimpica y RyalPurple para Horizonte http://en.wikibooks.org/wiki/LaTeX/Colors#The_68_standard_colors_known_to_dvips
\fancyhead[RE]{\textbf{laberintos e infinitos}}
\fancyhead[LO]{\textbf{En el Horizonte}} %La sección a la que pertenece el artículo
%\thispagestyle{plain}
\fancyfoot{}
\fancyfoot[C]{2.6570415588986394897683205211695427040366888}
\fancyfoot[LE,RO]{\thepage}
\parindent=0in
\hspace{1pt}
\textbf{\Large Nombre del artículo}
\begin{flushright}

Nombre del autor \\
\textit{Ocupaci\'on del autor}\\

\end{flushright}

%%%%%%%%%%%%%%%%%%%%%%%%%%%%%%%%%%%%%%%%%%%%%%%%%%%%%%%%%%%%%%%%%%%%%%%%%%%%%%%%
%%%%%%%%%%%%%%%%%%%%%%%%%%%%%%%%%%%%% EPÍGRAFE %%%%%%%%%%%%%%%%%%%%%%%%%%%%%%%%%
%%%%%%%%%%%%%%%%%%%%%%%%%%%%%%%%%%%%%%%%%%%%%%%%%%%%%%%%%%%%%%%%%%%%%%%%%%%%%%%%

\begin{flushright}
\textit{``Un ep\'igrafe es una cita que sintetiza o ilustra la idea general de un texto’’}\\
Pseudo manual de estilo de Laberintos e Infinitos\\
\end{flushright}





%%%%%%%%%%%%%%%%%%%%%%%%%%%%%%%%%%%%%%%%%%%%%%%%%%%%%%%%%%%%%%%%%%%%%%%%%%%%%%%%
%%%%%%%%%%%%%%%%%%%%%%%%%%%%%%%%%%%% SECCIONES %%%%%%%%%%%%%%%%%%%%%%%%%%%%%%%%%
%%%%%%%%%%%%%%%%%%%%%%%%%%%%%%%%%%%%%%%%%%%%%%%%%%%%%%%%%%%%%%%%%%%%%%%%%%%%%%%%

\section*{Nombre de la secci\'on.} %El asterisco es para que no aparezca el número de la sección.
\qquad S\'olo el primer p\'arrafo de la secci\'on lleva sangr\'ia. Esto incluye el primer p\'arrafo del art\'iculo. Tambi\'en podemos pones subsecciones y subsubsecciones.

\subsection*{Nombre de la subsecci\'on.}

\subsubsection*{Nombre de la subsubsecci\'on}






%%%%%%%%%%%%%%%%%%%%%%%%%%%%%%%%%%%%%%%%%%%%%%%%%%%%%%%%%%%%%%%%%%%%%%%%%%%%%%%%
%%%%%%%%%%%%%%%%%%%%%%%%%%%%% ESPACIO ENTRE PÁRAFOS %%%%%%%%%%%%%%%%%%%%%%%%%%%%
%%%%%%%%%%%%%%%%%%%%%%%%%%%%%%%%%%%%%%%%%%%%%%%%%%%%%%%%%%%%%%%%%%%%%%%%%%%%%%%%
Para terminar un p\'arrafo, no olvides poner dos diagonales hacia atr\'as ($\backslash \backslash$) y dejar un espacio antes de empezar el siguiente.\\

As\'i obtenemos los espacios entre p\'arrafos.\\




%%%%%%%%%%%%%%%%%%%%%%%%%%%%%%%%%%%%%%%%%%%%%%%%%%%%%%%%%%%%%%%%%%%%%%%%%%%%%%%%
%%%%%%%%%%%%%%%%%%%%%%%%%%% NOTAS AL PIE DE PÁGINA %%%%%%%%%%%%%%%%%%%%%%%%%%%%%
%%%%%%%%%%%%%%%%%%%%%%%%%%%%%%%%%%%%%%%%%%%%%%%%%%%%%%%%%%%%%%%%%%%%%%%%%%%%%%%%
Nota al pie.\footnote{Texto de la nota.} Notar que el superíndice va \textit{después} del punto (o el signo que sea). Es importante no poner bibliografía en las notas al pie de p\'agina.\\




%%%%%%%%%%%%%%%%%%%%%%%%%%%%%%%%%%%%%%%%%%%%%%%%%%%%%%%%%%%%%%%%%%%%%%%%%%%%%%%%%%%%%
%Definiciones, Teoremas, Lemas, Proposiciones, Corolarios, Ejemplos y Demostraciones%
%%%%%%%%%%%%%%%%%%%%%%%%%%%%%%%%%%%%%%%%%%%%%%%%%%%%%%%%%%%%%%%%%%%%%%%%%%%%%%%%%%%%%
\begin{definition}
Esta es una \textbf{definici\'on}. 
\end{definition}

\begin{theorem} %Cambiar por lemma, proposition, corollary o example según sea el caso
\textbf{Teorema del Valor Intermedio}\\
Texto del teorema.%Ignorar \textbf{} si el teorema no tiene nombre.
\end{theorem} %Debe coincidir con el \begin{} anterior.

\begin{proof}
Demostraci\'on del teorema.\\
\end{proof}




%%%%%%%%%%%%%%%%%%%%%%%%%%%%%%%%%%%%%%%%%%%%%%%%%%%%%%%%%%%%%%%%%%%%%%%%%%%%%%%%
%%%%%%%%%%%%%%%%%%%%%%%%%% CONSIDERACIONES DE ESTILO %%%%%%%%%%%%%%%%%%%%%%%%%%%
%%%%%%%%%%%%%%%%%%%%%%%%%%%%%%%%%%%%%%%%%%%%%%%%%%%%%%%%%%%%%%%%%%%%%%%%%%%%%%%%
\begin{enumerate}
\item Los \textbf{conceptos importantes} se pone en negritas la primera vez que aparecen en el art\'iculo.
\item Las comillas se ponen ``texto entrecomillado''.
\item Las \textit{letras cursivas} se usan cuando el autor las puso como \'enfasis, en t\'itulo de libros y en palabras en otro idioma.
\item No se usa texto subrayado.
\item No debe quedar una \'unica l\'inea de un p\'arrafo en una p\'agina si el resto del p\'arrafo est\'a en otra. 
\item No usamos $\backslash \backslash$ antes o despu\'es de los siguientes comandos:
\begin{itemize}
\item $\backslash$begin\{section\} (subsection,subsubsection).
\item $\backslash$begin\{center\}(flushright, flushleft, enumerate, itemize).
\item $\backslash$end\{center\}(flushright, flushleft, enumerate, itemize).
\item \$\$, $\backslash$[ o $\backslash$].
\end{itemize}
\item Si el art\'iculo requiere de numeraci\'on en secciones, definiciones, proposiciones, etc, s\'olo se requiere quitar los asteriscos en $\backslash$section*\{\} o en las l\'ineas 31-36 ($\backslash$newtheorem* $\ldots$)
\item Para poner elipsis se usa $\ldots$, para poner subraya se pone $\_$ y para obtener tilde se pone $\sim$.
\item Utilizamos el paquete \textit{listings} para poder importar y utilizar código de algún programa en el artículo. Por ejemplo, un breve programa de python se vería así: 
\begin{lstlisting}[language=Python]
parents, babies = (1, 1)
while babies < 100:
    print 'This generation has {0} babies'.format(babies)
    parents, babies = (babies, parents + babies)
\end{lstlisting}
\item La bibliografía sigue la guía de estilo del formato Chicago. 
\end{enumerate}



%%%%%%%%%%%%%%%%%%%%%%%%%%%%%%%%%%%%%%%%%%%%%%%%%%%%%%%%%%%%%%%%%%%%%%%%%%%%%%%%
%%%%%%%%%%%%%%%%%%%%%%%%%%%%%%%%%%%%% IMÁGENES %%%%%%%%%%%%%%%%%%%%%%%%%%%%%%%%%
%%%%%%%%%%%%%%%%%%%%%%%%%%%%%%%%%%%%%%%%%%%%%%%%%%%%%%%%%%%%%%%%%%%%%%%%%%%%%%%%
%La imagen debe estar en la misma carpeta que el artículo.
%\begin{center}
%\includegraphics[scale=0.23]{nombre del archivo.jpg}%cambiar el número en scale=0.23 para cambiar el tamaño de la imagen.   
%\end{center}









%%%%%%%%%%%%%%%%%%%%%%%%%%%%%%%%%%%%%%%%%%%%%%%%%%%%%%%%%%%%%%%%%%%%%%%%%%%%%%%%
%%%%%%%%%%%%%%%%%%%%%%%%%%%%%%%%%% REFERENCIAS %%%%%%%%%%%%%%%%%%%%%%%%%%%%%%%%%
%%%%%%%%%%%%%%%%%%%%%%%%%%%%%%%%%%%%%%%%%%%%%%%%%%%%%%%%%%%%%%%%%%%%%%%%%%%%%%%%
\begin{thebibliography}{3}

\bibitem{Libro} %Para libros
Autor, Nombre.
\textit{Nombre del libro}.
Editorial, año. 

\bibitem{Articulo} %Para artículos publicados en revistas.
Autor, Nombre.
``Título del artículo’’,
\textit{Nombre de la revista} número (año):
páginas.

\bibitem{PagWeb} %Para páginas web. Recordar usar $\_$ $\sim$ de ser necesario
``Nombre del artículo o página.’’
\textit{Wikipedia/Jstor (opcional)}.
Consultado el fecha de consulta.\\
\url{Direccion\_URL\_DeLaPagina}

\end{thebibliography}











%%%%%%%%%%%%%%%%%%%%%%%%%%%%%%%%%%%%%%%%%%%%%%%%%%%%%%%%%%%%%%%%%%%%%%%%%%%%%%%%%%%%%%

\end{document}